% 1D electric force example for RandAssign
% Geoffrey M. Poore, 2016
% License:  Creative Commons Zero (CC0) license
%           https://creativecommons.org/about/cc0
%
% Manual compile: run LaTeX, then PythonTeX, then LaTeX again
% Generate randomized assignments:  run `randassign example_electric_force_1d.tex`

\documentclass[11pt]{article}

\usepackage[margin=1in]{geometry}

\usepackage{siunitx}
\usepackage{pythontex}
\usepackage{pgfplots}
\usepackage{nopageno}


\title{Electric Field in 1D}
\author{\Large Name:  \underline{~~\input{name}~~} \\~\\ Attempt: \input{attempt}}
\date{}

\begin{document}

\maketitle

\hrule\vspace{0.02in}\hrule
~\linebreak~\linebreak

\noindent For each configuration, find the electric field at charge $q$ due to charges $Q_1$ and $Q_2$.  Then find the resulting force on $q$.


\begin{pycode}
from pylab import *
import random

from randassign import RandAssign
ra = RandAssign()

def make_problem():
    # Constants
    k =  8.9875517873681764e9
    
    # Functions for generating random values
    def rand_charge():
        return random.choice([-1, 1])*random.randrange(10, 100)*1e-6
    def rand_pos():
        return (random.randrange(-10, 10), 0)
    
    # Charge q parameters
    q = rand_charge()
    x, y = rand_pos()
    
    # Charge 1 parameters
    Q1 = rand_charge()
    while Q1 == q:
        Q1 = rand_charge()
    x1, y1 = rand_pos()
    while x1 == x:
        x1, y1 = rand_pos()
    
    # Charge 2 parameters
    Q2 = rand_charge()
    while Q2 == q or Q2 == Q1:
        Q2 = rand_charge()
    x2, y2 = rand_pos()
    while x2 == x or x2 == x1:
        x2, y2 = rand_pos()
    
    # The calculation below could be made simpler for a 1D case
    # The more general approach is used so that the code may be adapted to 2D
    
    # Calculate field from 1
    r1 = sqrt((x-x1)**2 + (y-y1)**2)
    E1 = k*abs(Q1)/r1**2
    theta1 = arctan2(sign(Q1)*(y-y1), sign(Q1)*(x-x1))
    E1x = E1*cos(theta1)
    E1y = E1*sin(theta1)
    
    # Calculate field from 2
    r2 = sqrt((x-x2)**2 + (y-y2)**2)
    E2 = k*abs(Q2)/r2**2
    theta2 = arctan2(sign(Q2)*(y-y2), sign(Q2)*(x-x2))
    E2x = E2*cos(theta2)
    E2y = E2*sin(theta2)
    
    # Calculate total field
    Ex = E1x + E2x
    Ey = E1y + E2y
    E = sqrt(Ex**2 + Ey**2)
    theta = arctan2(Ey, Ex)
    
    # Calculate the force due to the field
    F = abs(q)*E
    Fx = q*Ex
    Fy = q*Ey
    theta_F = arctan2(Fy, Fx)
    
    # Diagrams are created using the PGFPlots package for LaTeX
    # It would also be possible to use a Python plotting library
    problem_template = r'''
    \textbf{{Charges}}:
    $q=\SI{{{0:.3g}}}{{\micro\coulomb}}$,
    $Q_1=\SI{{{1:.3g}}}{{\micro\coulomb}}$, and
    $Q_2=\SI{{{2:.3g}}}{{\micro\coulomb}}$.
    
    \begin{{center}}
    \begin{{tikzpicture}}
    \begin{{axis}}[
        x=0.65cm, y=0.65cm,
        anchor=origin,
        at={{(0pt,0pt)}},
        disabledatascaling,
        xlabel=$x$ (m),
        ylabel=$y$ (m),
        xmin=-11, xmax=11, ymin=-2, ymax=2,
        grid=both]
    \draw[black,fill] ({3},{4}) circle (2pt) node [above] {{$q$}};
    \draw[black,fill] ({5},{6}) circle (2pt) node [above] {{$Q_1$}};
    \draw[black,fill] ({7},{8}) circle (2pt) node [above] {{$Q_2$}};
    \end{{axis}}
    \end{{tikzpicture}}
    \end{{center}}
    '''
    
    problem = problem_template.format(q/1e-6, Q1/1e-6, Q2/1e-6, x, y, x1, y1, x2, y2)
    
    # For a 1D case, want the x component of the field and force
    return (problem, Ex, Fx)

# Generate 4 unique problems
# The chance of two identical problems is very low in this case
# But checking for identical problems won't hurt
problems = []
while len(problems) < 4:
    p, e, f = make_problem()
    if p not in problems:
        problems.append(p)
        t = '\\SI{{{0:.3g}}}{{\\newton/\\coulomb}} and \\SI{{{1:.3g}}}{{\\newton}}'
        ra.addsoln(t.format(e, f))

print('\\begin{enumerate}')
for n, p in enumerate(problems):
    print('\\item ' + p)
    if n+1 < len(problems):
        print(r'\vspace{1in}')
print('\\end{enumerate}')
\end{pycode}

\end{document}
