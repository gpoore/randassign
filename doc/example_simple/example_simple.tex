% Simple example for RandAssign
% Geoffrey M. Poore, 2015
% License:  Creative Commons Zero (CC0) license
%           https://creativecommons.org/about/cc0
%
% Manual compile: run LaTeX, then PythonTeX, then LaTeX again
% Generate randomized assignments:  run `randassign example_simple.tex`

\documentclass{article}

\usepackage{nopageno}
\usepackage{pythontex}
\usepackage{xcolor}
\usepackage{tikz}

\begin{pycode}
from randassign import RandAssign
ra = RandAssign()
from math import *
import random
\end{pycode}

\title{Simple Randomized Assignment \\ {\Large Attempt: \input{attempt.tex}}}
\author{Name: \underline{~~~~\input{name.tex}~~~~}}

\begin{document}

\maketitle

\begin{pycode}
a = random.randint(1, 10)
b = random.randint(1, 10)
c = sqrt(a**2 + b**2)

drawing = '''
\\begin{{tikzpicture}}
\\draw[style=help lines] (0,0) grid (10,10);
\\filldraw[fill=blue!40!white] (0,0) -- ({a},0) -- ({a},{b}) -- cycle;
\\end{{tikzpicture}}
'''.format(a=a, b=b)

ra.addsoln(round(c, 2))
\end{pycode}

A right triangle has a base of length \py{a} and a height of length \py{b}.  Find the length of the hypotenuse.
\begin{center}
\py{drawing}
\end{center}

\end{document}
